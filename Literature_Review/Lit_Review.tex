%Style
\documentclass[12pt]{article}
\usepackage[top=1in, bottom=1in, left=1in, right=1in]{geometry}
\parindent 22pt
\usepackage{fancyhdr}

%Packages
\usepackage{adjustbox}
\usepackage{amsmath}
\usepackage{amsfonts}
\usepackage{amssymb}
\usepackage[english]{babel}
\usepackage{bm}
\usepackage[table]{xcolor}
\usepackage{tabu}
\usepackage{color,soul}
\usepackage[utf8x]{inputenc}
\usepackage{makecell}
\usepackage{longtable}
\usepackage{multirow}
\usepackage[normalem]{ulem}
\usepackage{etoolbox}
\usepackage{graphicx}
\usepackage{tabularx}
\usepackage{ragged2e}
\usepackage{booktabs}
\usepackage{caption}
\usepackage{fixltx2e}
\usepackage[para, flushleft]{threeparttablex}
\usepackage[capposition=top]{floatrow}
\usepackage{subcaption}
\usepackage{pdfpages}
\usepackage{pdflscape}
\usepackage[sort&compress]{natbib}
\usepackage{bibunits}
\usepackage[colorlinks=true,linkcolor=darkgray,citecolor=darkgray,urlcolor=darkgray,anchorcolor=darkgray]{hyperref}
\usepackage{marvosym}
\usepackage{makeidx}
\usepackage{setspace}
\doublespacing
\usepackage{enumerate}
\usepackage{rotating}
\usepackage{epstopdf}
\usepackage[titletoc]{appendix}
\usepackage{framed}
\usepackage{comment}
\usepackage{xr}
\usepackage{titlesec}
\usepackage{footnote}
\usepackage{longtable}
\newlength{\tablewidth}
\setlength{\tablewidth}{9.3in}
\usepackage[bottom]{footmisc}
\usepackage{stackengine}
\newcommand\barbelow[1]{\stackunder[1.2pt]{$#1$}{\rule{1ex}{.085ex}}}
\usepackage{titletoc}
\usepackage{accents}
\usepackage{arydshln }
\usepackage{titletoc}
\titlespacing{\section}{.2pt}{1ex}{1ex}
\setcounter{section}{0}
\renewcommand{\thesection}{\arabic{section}}


\makeatletter
\pretocmd\start@align
{%
  \let\everycr\CT@everycr
  \CT@start
}{}{}
\apptocmd{\endalign}{\CT@end}{}{}
\makeatother
%Watermark
\usepackage[printwatermark]{xwatermark}
\usepackage{lipsum}
\definecolor{lightgray}{RGB}{220,220,220}
\definecolor{dimgray}{RGB}{105,105,105}

%\newwatermark[allpages,color=lightgray,angle=45,scale=3,xpos=0,ypos=0]{Preliminary Draft}

%Further subsection level
\usepackage{titlesec}
\titleformat{\paragraph}
{\normalfont\normalsize\bfseries}{\theparagraph}{1em}{}
\titlespacing*{\paragraph}
{0pt}{3.25ex plus 1ex minus .2ex}{1.5ex plus .2ex}

\titleformat{\subparagraph}
{\normalfont\normalsize\bfseries}{\thesubparagraph}{1em}{}
\titlespacing*{\subparagraph}
{0pt}{3.25ex plus 1ex minus .2ex}{1.5ex plus .2ex}

%Functions
\DeclareMathOperator{\cov}{Cov}
\DeclareMathOperator{\sign}{sgn}
\DeclareMathOperator{\var}{Var}
\DeclareMathOperator{\plim}{plim}
\DeclareMathOperator*{\argmin}{arg\,min}
\DeclareMathOperator*{\argmax}{arg\,max}

%Math Environments
\usepackage{amsthm}
\newtheoremstyle{mytheoremstyle} % name
    {\topsep}                    % Space above
    {\topsep}                    % Space below
    {\color{black}}                   % Body font
    {}                           % Indent amount
    {\itshape \color{dimgray}}                   % Theorem head font
    {.}                          % Punctuation after theorem head
    {.5em}                       % Space after theorem head
    {}  % Theorem head spec (can be left empty, meaning ?normal?)

\theoremstyle{mytheoremstyle}
\newtheorem{assumption}{Assumption}
\renewcommand\theassumption{\arabic{assumption}}

\theoremstyle{mytheoremstyle}
\newtheorem{assumptiona}{Assumption}
\renewcommand\theassumptiona{\arabic{assumptiona}a}

\newtheorem{assumptionb}{Assumption}
\renewcommand\theassumptionb{\arabic{assumptionb}b}

\newtheorem{assumptionc}{Assumption}
\renewcommand\theassumptionc{\arabic{assumptionc}c}

\theoremstyle{mytheoremstyle}
\newtheorem{lemma}{Lemma}

\theoremstyle{mytheoremstyle}
\newtheorem{proposition}{Proposition}

\theoremstyle{mytheoremstyle}
\newtheorem{corollary}{Corollary}

%Commands
\newcommand\independent{\protect\mathpalette{\protect\independenT}{\perp}}
\def\independenT#1#2{\mathrel{\rlap{$#1#2$}\mkern2mu{#1#2}}}
\newcommand{\overbar}[1]{\mkern 1.5mu\overline{\mkern-1.5mu#1\mkern-1.5mu}\mkern 1.5mu}
\newcommand{\equald}{\ensuremath{\overset{d}{=}}}
\captionsetup[table]{skip=10pt}
%\makeindex

%Table, Figure, and Section Styles
\captionsetup[figure]{labelfont={bf},name={Figure},labelsep=period}
\renewcommand{\thefigure}{\arabic{figure}}
\captionsetup[table]{labelfont={bf},name={Table},labelsep=period}
\renewcommand{\thetable}{\arabic{table}}
\titleformat{\section}{\centering \normalsize \bf}{\thesection.}{0em}{}%\titlespacing*{\subsection}{0pt}{0\baselineskip}{0\baselineskip}
\renewcommand{\thesection}{\arabic{section}}

\titleformat{\subsection}{\flushleft \normalsize \bf}{\thesubsection}{0em}{}
\renewcommand{\thesubsection}{\arabic{section}.\arabic{subsection}}

%No indent
\setlength\parindent{24pt}
\setlength{\parskip}{5pt}

%Logo
%\AddToShipoutPictureBG{%
%  \AtPageUpperLeft{\raisebox{-\height}{\includegraphics[width=1.5cm]{uchicago.png}}}
%}

\newcolumntype{L}[1]{>{\raggedright\let\newline\\\arraybackslash\hspace{0pt}}m{#1}}
\newcolumntype{C}[1]{>{\centering\let\newline\\\arraybackslash\hspace{0pt}}m{#1}}
\newcolumntype{R}[1]{>{\raggedleft\let\newline\\\arraybackslash\hspace{0pt}}m{#1}} 

\newcommand{\mr}{\multirow}
\newcommand{\mc}{\multicolumn}

%\newcommand{\comment}[1]{}


% Title
\begin{document}
\begin{singlespace}
\title{Literature Review}
\author{Austyn West\thanks{Department of Economics, Texas A\&M University.}}
\date{October 16, 2025}
\maketitle
\end{singlespace}


\section*{Literature Review}

The literature on broadband infrastructure and educational outcomes is broad and interdisciplinary, spanning work in education economics, public finance, and information technology policy. Three areas are particularly relevant to the 2014 E-Rate modernization: (i) the effects of digital access and technology adoption on human capital formation, (ii) the role of infrastructure investments in school environments and learning outcomes, and (iii) the persistence of the digital divide and heterogeneity in broadband access.

\textbf{Technology, Connectivity, and Human Capital Formation.} The role of education in human capital formation is a cornerstone of economic theory. \citet{mincer1974} established the Mincer equation, showing that each additional year of schooling increases individual earnings by roughly 5–8\%, demonstrating the microeconomic returns to education. \citet{hanushek2012} provide complementary macro-level evidence, finding that improvements in cognitive skills are associated with a 2 percentage point higher annual GDP growth rate across nations over 40 years. Closely related broadband research, such as \citet{hazlett2019}, indicates that basic connectivity alone had limited effects on standardized test scores, suggesting that technology must be effectively integrated to impact human capital. Together, these studies highlight both the magnitude of returns to education and the potential, but conditional, role of digital infrastructure in enhancing learning outcomes.

\textit{Broadband Access.}  Research shows that broadband availability can shape educational opportunities through development of skills and enhanced learning activities. Studies such as \citet{hazlett2019}, \citet{dettling2015}, and \citet{niewoehner2025} suggest that connectivity influences learning environments, though limited improvements in SAT scores prior to 2014 \citep{hazlett2019} highlight the constraints of basic access. Earlier work by \citet{angrist2002}, \citet{angrist2009}, \citet{beuermann2013}, and \citet{duflo2012} highlights how interventions targeting teachers can improve student outcomes. For example, \citet{duflo2012} show that increasing teacher attendance through incentive programs raised student test scores by 0.17 standard deviations after one year. While Duflo et al. focus on direct incentives, these findings conceptually illustrate that enhancing teacher effectiveness can boost student human capital. In the context of broadband, providing teachers with reliable high-speed internet and digital tools may similarly increase their capacity to deliver high-quality instruction, thereby amplifying student learning outcomes.

\textit{Broadband Speed.} The quality of broadband, measured by speed, also affects educational delivery. Studies examining broadband quality show that higher-speed connections facilitate advanced learning tools. \citet{sanchis2021} report that a one-standard deviation increase in broadband speed raises test scores by 0.19 standard deviations, comparable in magnitude to teacher incentive effects from \citet{duflo2012}. \citet{boeri2023} and \citet{grimes2018} further emphasize that the effectiveness of broadband depends on socioeconomic and infrastructural conditions, indicating that speed alone is insufficient to uniformly enhance educational outcomes.

The pre-reform focus on connectivity \citep{hazlett2019} and the limited post-reform evaluation leave a significant gap in understanding the dynamic relationship between broadband technologies and human capital development across diverse socioeconomic and geographic contexts. This study addresses this gap by estimating the causal impact of the 2014 E-Rate modernization on graduation rates, college enrollment, and earnings, investigating how enhanced access and speed collectively reshape human capital formation. 

\textbf{Infrastructure Investments in School Environments and Learning Outcomes.}  The influence of school infrastructure on educational outcomes has long been a focus of empirical research. Early studies by \citet{berner1993}, \citet{cash1993}, and \citet{duran2008} establish that physical conditions, such as building quality and classroom environment, play a pivotal role in student performance and attendance with \citet{berner1993} finding a positive association between building condition improvements and student achievement.  Subsequent work by \citet{maxwell1999}, \citet{hathaway1995}, and \citet{park2020} extends this to environmental factors like lighting and temperature. A study by \citet{park2020} finds that without air conditioning, a one degree hotter school year reduces that year’s learning by one percent. More recently, \citet{wanke2024} and \citet{daoud2021} emphasize the growing importance of technological infrastructure, including broadband and home internet, as mediators of educational success, with \citet{wanke2024} explaining 30–40\% of performance variance through infrastructure and teacher capital. Together, these studies suggest that investments in both physical and technological assets can enhance educational quality through giving students access to quality broadband infrastructure and potentially increasing the human capital of teachers.

This study contributes by evaluating how the 2014 E-Rate modernization’s focus on broadband and Wi-Fi infrastructure impacts attainment outcomes, building on existing infrastructure research to examine the impact of broadband infrastructure and quality on educational outcomes.

\textbf{The Digital Divide and Heterogeneity in Broadband Access.}  Disparities in broadband access and quality across geographic and demographic lines persist. \citet{gallardo2022} and \citet{gallardo2024} highlights continuing rural–urban and socioeconomic gaps, with \citet{gallardo2024} noting that as the share of the rural population, the population age 65 or older, and individual poverty rates increased, the average download speed decreased. Whereas, \citet{li2023} and \citet{reddick2020} identify racial and income-based disparities often tied to affordability and infrastructure deficits, with \citet{li2023} finding that broadband access in majority Black and Hispanic neighborhoods was 10–15\% lower than in majority White or Asian neighborhoods. \citet{reddick2020} shows that lower-income and minority groups are more likely to have basic connectivity but rarely benefit from high-speed options due to affordability barriers. Historical studies (\citet{strover2001}, \citet{riddlesden2014},  \citet{kelley2020}) trace the evolution of these disparities in the U.S. and internationally. \citet{boeri2023} further notes that wealthier communities disproportionately access higher-speed broadband. 

This study advances the literature by analyzing how the 2014 E-Rate reforms mitigate digital disparities and contribute to attainment gains, particularly in underserved regions, aligning with the proposal’s emphasis on equity and rural development.

\bibliographystyle{chicago}
\bibliography{Bibliography}

\end{document}