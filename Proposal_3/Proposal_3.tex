\input{preamble}

% Title
\begin{document}
\begin{singlespace}
\title{Proposal 3}
\author{Austyn West\thanks{Department of Economics, Texas A\&M University.}}
\date{October 2, 2025}
\maketitle
\end{singlespace}

\textbf{Research Question.}  The 2014 E-Rate modernization order increased rural broadband and infrastructure funding by 60\%, from \$2.4 to \$3.9 billion annually \citep{federalcommunicationscommisionSummarySecondERate}. In rural Appalachia, 86.2\% of households had broadband compared to 89.7\% nationally, and 116 counties remained below 80\% as of 2023 \citep{srygleyAPPALACHIANREGIONDATA}. Evidence on broadband’s educational impact is mixed. For example, \cite{hazlettEducationalImpactBroadband2019} found no E-Rate effect on SAT scores in North Carolina from 2000–2013 (pre-modernization), citing limited broadband focus at the time. Similarly, \cite{boeriHighSpeedBroadbandSchool2023} found no significant broadband effect on Italian student performance (2012–2019), though there is socioeconomic heterogeneity. This proposal asks: What is the effect of the 2014 E-Rate reforms on educational attainment such as graduation, college enrollment, and earnings in rural Appalachia? Given the region’s low broadband and computer access alongside low educational outcomes, the research will provide insight on post-modernization broadband and connectivity investments in human capital.

\textbf{Economic Framework.} The expectation is that the post-2014 E-Rate shift to high-speed broadband (labeled category 1) and internal infrastructure (labeled category 2, e.g., Wi-Fi) reduces barriers, enhancing educational and economic outcomes. Improved school digital infrastructure may boost graduation rates through online learning and college applications. \cite{dettlingEveryLittleBit2015} found that students in zip codes with broadband outperform their academic peers in zip codes without broadband by an average of 0.7 SAT point.\footnote{Equivalent to 0.3 percent of a standard deviation.} The model would show diminishing returns or larger gains in low-connectivity areas, but selection bias emerges, as high-poverty counties can receive discounts (up to 90\%)\footnote{Discount rates are based on the percentage of students eligible for the National School Lunch Program. See the FCC’s E-Rate discount matrix: \url{https://www.usac.org/wp-content/uploads/e-rate/documents/samples/Discount-Matrix.pdf}.} leading to self-selection for funding.  

\textbf{Empirical Design.} This study applies the Difference-in-Discontinuities (DiDisc) approach of \citet{grembiFiscalRulesMatter2016}, which compares the discontinuity in outcomes at the FCC discount threshold before and after the 2014 reform. The general specification is:
\[
Y_{it} = \beta_0 + \beta_1 \text{Post}_t + \beta_2 \text{Above}_i + \beta_3 (\text{Post}_t \times \text{Above}_i) 
+ f(X_{it}) + f(X_{it}) \cdot \text{Post}_t + f(X_{it}) \cdot \text{Above}_i 
+ \mu_i + \tau_t + \epsilon_{it},
\]
where $Y_{it}$ is graduation, college enrollment, or earnings; $\text{Post}_t$ indicates post-2014 years; $\text{Above}_i$ marks counties above the cutoff (e.g., 75\% NSLP eligibility); and $X_{it}$ is the running variable. The flexible functions $f(X_{it})$ (e.g., local linear or polynomial) capture heterogeneous trends across time and sides of the cutoff. The coefficient of interest, $\beta_3$, measures the difference in discontinuities, while county and year fixed effects $\mu_i$ and $\tau_t$ absorb time-invariant characteristics and common shocks. This design identifies local causal effects of the E-Rate moderniation for counties near the threshold.


\textbf{Data.} County-year observations from 2010–2019, supporting DiD with pre- and post-modernization periods. E-Rate funding (USAC, 2010–2019), graduation rates (NCES CCD), college enrollment (ACS 5-year estimates), earnings (Opportunity Atlas), and ARC county data. Focus is on the 116 counties below 80\% broadband, consistent with low-connectivity emphasis in the framework. Key measures include E-Rate funding (\$ per pupil), graduation rates, college enrollment, earnings, and controls (poverty, teacher ratio).
\bibliographystyle{chicago}
\bibliography{Bibliography}



\end{document}

