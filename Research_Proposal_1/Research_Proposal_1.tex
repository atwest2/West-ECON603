\input{preamble}
% Title
\title{Proposal 1}
\author{Austyn West\thanks{Department of Economics, Texas A\&M University.} }
\date{\today}

\begin{document}
\maketitle

\section*{Research Question}
The opioid epidemic has severely affected the Appalachian region, with serious implications for families within the region. Appalachia spans 428 counties across 13 states, home to over 26 million people, of whom 50.2 percent are aged 25–64 and 78.7 percent are predominantly white non-Hispanic \citep{srygleyAPPALACHIANREGIONDATA}. The region’s economy has historically relied on one industry, coal mining, but employment in this industry has plummeted to 62 percent of its 2000 level, with Central Appalachia experiencing an even steeper decline to 54 percent \citep{CoalProductionEmployment}. This economic downturn has potentially fueled opioid misuse, as evidenced by \citet{hollingsworthMacroeconomicConditionsOpioid2017}, who found that in the U.S. a one standard deviation increase in the unemployment rate correlates with a 9 percent rise in fatal opioid overdoses. In certain Appalachian counties, OxyContin prescription rates were reported to be five to six times the national average \citep{moodySubstanceUseRural2017}, while mortality among white non-Hispanics rose dramatically during the opioid crisis, increasing by 34 deaths per 100,000 between 1999 and 2013 \citep{caseRisingMorbidityMortality2015}. For children in Appalachia, parental opioid misuse and overdose can create downstream adverse effects on their education. Chronic absenteeism in particular is shown to cause harm in academic outcomes. Research by \citet{rhoad-drogalisAbsenteeismAppalachianPreschool2018} indicates that while preschool absences of 5 to 7.5 percent of the school year have minimal impact, absences reaching 10 percent lead to significant academic setbacks. On average, Appalachian preschoolers miss 13.1 percent of the school year, a rate that may be partly driven by parental opioid use. Such persistent disruptions could undermine long-term academic achievement and graduation rates. This proposal seeks to address the question: How does parental opioid exposure (e.g. misuse, overdose, or incarceration) affect their children's educational outcomes in Appalachian schools? Looking at a region which has been hit with significant job lose in its dominant industry and high rates of opioid misuse, this research can hopefully identify mechanisms through which the opioid crisis is affecting future generations of worker's educational attainment.

\section*{Economic Framework and Empirical Design}
Building on the context described above, the following framework is used to examine, in a plausibly causal way, how the decline of coal employment in Appalachia may affect children’s schooling through a clear pathway. As coal jobs disappear, households experience economic stress and instability, increasing the likelihood that parents are either exposed to opioids or increase current use, potentially leading to misuse, overdose, or incarceration. In these circumstances, parents’ health and income are expected to decline, which may reduce the time, attention, and supervision they can devote to their children’s learning. Children facing this instability often struggle in school, showing up less frequently, putting in less effort, and falling behind grade-level expectations.  The proposed hypothesis is the following: job loss increases parental opioid exposure, which diminishes children’s educational inputs and translates into lower attendance, weaker test performance, and reduced likelihood of on-time graduation. To investigate these effects, the study would take advantage of variation in coal mine closures as a source of plausibly exogenous employment shocks. The design proceeds in two main steps. First, coal mine closures are used to measure county-level job losses that are largely independent of local school quality or demand for education. These job losses are then linked to opioid overdoses and, in turn, to children’s outcomes, focusing on absenteeism as the primary channel through which household instability affects learning, while also considering test scores and graduation as downstream consequences. A complementary approach compares schools in counties affected by closures to those less affected before and after major closure events, helping to isolate the impact of local economic shocks from broader trends. By identifying these causal links, the study could provide a foundation for possible policy discussions, highlighting where targeted interventions or support programs could mitigate the educational consequences of opioid exposure in economically distressed communities.
\section*{Data}
This study would rely on a combination of publicly available and restricted-use data from 2009-2019. Public sources include records of coal mine closures from the Mine Safety and Health Administration, county-level employment and demographic information from the Bureau of Labor Statistics and the American Community Survey. Educational outcomes, including absenteeism, test scores, and graduation rates are potentially drawn from the National Center for Education Statistics Common Core of Data. Restricted-use datasets, such as treatment admissions from Substance Abuse and Mental Health Services Administration, student earnings information from the Institute of Education Sciences and/or Census, and overdose data from CDC's Wide-ranging Online Data for Epidemiologic Research may also be included after obtaining access. The analysis would focus on Central Appalachia, where coal dependence has historically been highest and the effects of economic disruption and opioid exposure are likely to be strongest. County-level data allow for the estimation of how closures and subsequent job losses influence opioid misuse and absenteeism, while school-level data support comparisons across affected and less-affected areas. Key variables include coal mine closures, employment changes, overdose rates, absenteeism, test scores, and graduation outcomes, as well as measures of local poverty and demographic composition. Using all of the sources described above, the study can hopefully ascertain the effects of economic shocks facing coal-dependent Appalachia and children's educational outcomes, identifying both the scope and magnitude of the opioid's crisis impact on children's educations and futures.
\bibliographystyle{chicago}
\bibliography{Bibliography} 
\end{document}


The opioid epidemic has severely impacted the Appalachian region for decades, with profound consequences for local families. Appalachia spans 428 counties across 13 states, home to over 26 million people, of whom 50.2 percent are aged 25–64 and 78.7 percent are predominantly white non-Hispanic \citep{srygleyAPPALACHIANREGIONDATA}. The region’s economy has historically relied on coal mining, but employment in this industry has plummeted to 62 percent of its 2000 level, with Central Appalachia experiencing an even steeper decline to 54 percent \citep{CoalProductionEmployment}. This economic downturn has fueled opioid misuse, as evidenced by \citet{hollingsworthMacroeconomicConditionsOpioid2017}, who found that a one standard deviation increase in the unemployment rate correlates with a 9 percent rise in fatal opioid overdoses. In certain Appalachian counties, OxyContin prescription rates have reached five to six times the national average \citep{moodySubstanceUseRural2017}. Concurrently, mortality rates among white non-Hispanics surged during the opioid crisis, increasing by 34 deaths per 100,000 from 1999 to 2013 \citep{caseRisingMorbidityMortality2015}. For children in Appalachia, parental opioid misuse, overdose, or incarceration can disrupt their education, with chronic absenteeism serving as a critical lens for understanding these effects. Research by \citet{rhoad-drogalisAbsenteeismAppalachianPreschool2018} indicates that while preschool absences of 5 to 7.5 percent of the school year have minimal impact, absences reaching 10 percent lead to significant academic setbacks. On average, Appalachian preschoolers miss 13.1 percent of the school year, a rate that may be partly driven by parental opioid use. Such persistent disruptions could undermine long-term academic achievement and graduation rates.This proposal investigates the question: How does parental opioid exposure (e.g., misuse, overdose, or incarceration) affect children’s educational outcomes in Appalachian schools? By examining a region marked by significant job losses in its dominant industry and elevated opioid misuse, this research aims to uncover the mechanisms through which the opioid crisis shapes the educational attainment of future generations.