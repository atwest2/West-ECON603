%Style
\documentclass[12pt]{article}
\usepackage[top=1in, bottom=1in, left=1in, right=1in]{geometry}
\parindent 22pt
\usepackage{fancyhdr}

%Packages
\usepackage{adjustbox}
\usepackage{amsmath}
\usepackage{amsfonts}
\usepackage{amssymb}
\usepackage[english]{babel}
\usepackage{bm}
\usepackage[table]{xcolor}
\usepackage{tabu}
\usepackage{color,soul}
\usepackage[utf8x]{inputenc}
\usepackage{makecell}
\usepackage{longtable}
\usepackage{multirow}
\usepackage[normalem]{ulem}
\usepackage{etoolbox}
\usepackage{graphicx}
\usepackage{tabularx}
\usepackage{ragged2e}
\usepackage{booktabs}
\usepackage{caption}
\usepackage{fixltx2e}
\usepackage[para, flushleft]{threeparttablex}
\usepackage[capposition=top]{floatrow}
\usepackage{subcaption}
\usepackage{pdfpages}
\usepackage{pdflscape}
\usepackage[sort&compress]{natbib}
\usepackage{bibunits}
\usepackage[colorlinks=true,linkcolor=darkgray,citecolor=darkgray,urlcolor=darkgray,anchorcolor=darkgray]{hyperref}
\usepackage{marvosym}
\usepackage{makeidx}
\usepackage{setspace}
\doublespacing
\usepackage{enumerate}
\usepackage{rotating}
\usepackage{epstopdf}
\usepackage[titletoc]{appendix}
\usepackage{framed}
\usepackage{comment}
\usepackage{xr}
\usepackage{titlesec}
\usepackage{footnote}
\usepackage{longtable}
\newlength{\tablewidth}
\setlength{\tablewidth}{9.3in}
\usepackage[bottom]{footmisc}
\usepackage{stackengine}
\newcommand\barbelow[1]{\stackunder[1.2pt]{$#1$}{\rule{1ex}{.085ex}}}
\usepackage{titletoc}
\usepackage{accents}
\usepackage{arydshln }
\usepackage{titletoc}
\titlespacing{\section}{.2pt}{1ex}{1ex}
\setcounter{section}{0}
\renewcommand{\thesection}{\arabic{section}}


\makeatletter
\pretocmd\start@align
{%
  \let\everycr\CT@everycr
  \CT@start
}{}{}
\apptocmd{\endalign}{\CT@end}{}{}
\makeatother
%Watermark
\usepackage[printwatermark]{xwatermark}
\usepackage{lipsum}
\definecolor{lightgray}{RGB}{220,220,220}
\definecolor{dimgray}{RGB}{105,105,105}

%\newwatermark[allpages,color=lightgray,angle=45,scale=3,xpos=0,ypos=0]{Preliminary Draft}

%Further subsection level
\usepackage{titlesec}
\titleformat{\paragraph}
{\normalfont\normalsize\bfseries}{\theparagraph}{1em}{}
\titlespacing*{\paragraph}
{0pt}{3.25ex plus 1ex minus .2ex}{1.5ex plus .2ex}

\titleformat{\subparagraph}
{\normalfont\normalsize\bfseries}{\thesubparagraph}{1em}{}
\titlespacing*{\subparagraph}
{0pt}{3.25ex plus 1ex minus .2ex}{1.5ex plus .2ex}

%Functions
\DeclareMathOperator{\cov}{Cov}
\DeclareMathOperator{\sign}{sgn}
\DeclareMathOperator{\var}{Var}
\DeclareMathOperator{\plim}{plim}
\DeclareMathOperator*{\argmin}{arg\,min}
\DeclareMathOperator*{\argmax}{arg\,max}

%Math Environments
\usepackage{amsthm}
\newtheoremstyle{mytheoremstyle} % name
    {\topsep}                    % Space above
    {\topsep}                    % Space below
    {\color{black}}                   % Body font
    {}                           % Indent amount
    {\itshape \color{dimgray}}                   % Theorem head font
    {.}                          % Punctuation after theorem head
    {.5em}                       % Space after theorem head
    {}  % Theorem head spec (can be left empty, meaning ?normal?)

\theoremstyle{mytheoremstyle}
\newtheorem{assumption}{Assumption}
\renewcommand\theassumption{\arabic{assumption}}

\theoremstyle{mytheoremstyle}
\newtheorem{assumptiona}{Assumption}
\renewcommand\theassumptiona{\arabic{assumptiona}a}

\newtheorem{assumptionb}{Assumption}
\renewcommand\theassumptionb{\arabic{assumptionb}b}

\newtheorem{assumptionc}{Assumption}
\renewcommand\theassumptionc{\arabic{assumptionc}c}

\theoremstyle{mytheoremstyle}
\newtheorem{lemma}{Lemma}

\theoremstyle{mytheoremstyle}
\newtheorem{proposition}{Proposition}

\theoremstyle{mytheoremstyle}
\newtheorem{corollary}{Corollary}

%Commands
\newcommand\independent{\protect\mathpalette{\protect\independenT}{\perp}}
\def\independenT#1#2{\mathrel{\rlap{$#1#2$}\mkern2mu{#1#2}}}
\newcommand{\overbar}[1]{\mkern 1.5mu\overline{\mkern-1.5mu#1\mkern-1.5mu}\mkern 1.5mu}
\newcommand{\equald}{\ensuremath{\overset{d}{=}}}
\captionsetup[table]{skip=10pt}
%\makeindex

%Table, Figure, and Section Styles
\captionsetup[figure]{labelfont={bf},name={Figure},labelsep=period}
\renewcommand{\thefigure}{\arabic{figure}}
\captionsetup[table]{labelfont={bf},name={Table},labelsep=period}
\renewcommand{\thetable}{\arabic{table}}
\titleformat{\section}{\centering \normalsize \bf}{\thesection.}{0em}{}%\titlespacing*{\subsection}{0pt}{0\baselineskip}{0\baselineskip}
\renewcommand{\thesection}{\arabic{section}}

\titleformat{\subsection}{\flushleft \normalsize \bf}{\thesubsection}{0em}{}
\renewcommand{\thesubsection}{\arabic{section}.\arabic{subsection}}

%No indent
\setlength\parindent{24pt}
\setlength{\parskip}{5pt}

%Logo
%\AddToShipoutPictureBG{%
%  \AtPageUpperLeft{\raisebox{-\height}{\includegraphics[width=1.5cm]{uchicago.png}}}
%}

\newcolumntype{L}[1]{>{\raggedright\let\newline\\\arraybackslash\hspace{0pt}}m{#1}}
\newcolumntype{C}[1]{>{\centering\let\newline\\\arraybackslash\hspace{0pt}}m{#1}}
\newcolumntype{R}[1]{>{\raggedleft\let\newline\\\arraybackslash\hspace{0pt}}m{#1}} 

\newcommand{\mr}{\multirow}
\newcommand{\mc}{\multicolumn}

%\newcommand{\comment}[1]{}


\begin{document}
\begin{singlespace}
\title{Data Section}
\author{Austyn West\thanks{Department of Economics, Texas A\&M University.}}
\date{October 30, 2025}
\maketitle
\end{singlespace}

\section*{Data}

\textbf{Data Sources:} We combine data from two primary sources. The first source is the Texas Academic Performance Reports (TAPR), published annually by the Texas Education Agency (TEA). The dataset covers the 2012--13 through 2018--19 school years and provides information on student performance, demographics, attendance, and college readiness for Texas public schools. The second source is the E-Rate data provided by the Universal Service Administrative Company (USAC), specifically Service Provider Invoice Forms (Form 472) and Billed Entity Applicant Reimbursement Forms (Form 474). These datasets contain school-level records on requested and approved funding for telecommunication and technology services, the discount rates applied, and the amounts disbursed under the federal E-Rate program.\footnote{Currently, USAC publicly reports Forms 472 and 474 data only for approvals granted in 2016 and later. A Freedom of Information Act (FOIA) request will be filed to obtain pre-2016 data once the federal government resumes normal operations.}

\textbf{Contents of the Data Sets:} TAPR reports disaggregated outcomes for several student subgroups based on race, gender, and socioeconomic status. The data include student counts, academic performance indicators, attendance rates, dropout rates, and college readiness metrics. E-Rate data includes each school's funding requests, vendor information, approved amounts, and discount percentages.\footnote{The Universal Service Administrative Company (USAC) and the Federal Communications Commission (FCC) require an upfront competitive bidding period of at least 28 days for both Category One and Category Two services.} E-Rate funding is divided into two service categories: Category One, which covers data transmission services and internet access, and Category Two, which covers internal connections (IC), managed internal broadband services (MIBS), and basic maintenance of internal connections (BMIC). Discounts are determined by the percentage of students in each school district which are eligible for the National School Lunch Program (NSLP) and by rural or urban location. The highest tier, districts with at least 75\% of students eligible for NSLP, receive a 90\% discount for Category One services and an 85\% discount for Category Two services.

\textbf{Sample Restrictions and Unit of Analysis:} The unit of analysis is the individual school-year. The sample includes all public schools in Texas that appear in TAPR for every year from 2012--13 through 2018--19. To ensure data consistency and minimize masking bias, multiple exclusions are applied. First, all 1A schools, defined as campuses with fewer than 105 enrolled students, are excluded because reported subgroup data are suppressed when the count of students in a category is five or fewer. Second, juvenile justice, disciplinary alternative, and charter schools with non-standard grading structures are excluded due to incomparable testing regimes.

\textbf{Key Variables and Outcomes:} Primary outcomes include measures of academic performance and college readiness. For elementary and middle schools, we use indicators from the Student Success Initiative (SSI), which reports grade-specific results for the STAAR Reading and Mathematics assessments in grades 5 and 8.\footnote{The 2012--13 school year was the first year the STAAR system replaced the previous Texas Assessment of Knowledge and Skills (TAKS).} In earlier years, two indicators are available: (i) Index 1: Performance, Level II Phase 1, which captures the share of students meeting the minimum satisfactory standard, and (ii) Students Requiring Accelerated Instruction, which identifies the proportion of students below the passing threshold mandated for remediation.\footnote{Students failing to meet the passing standard on STAAR Reading or Mathematics are required to receive accelerated instruction per section 28.0211 of the Texas Education Code and may be retained if not promoted by a Grade Placement Committee (GPC).} The first indicator can be interpreted as analogous to the later “Meeting Approaches Grade Level” measure, while the second indicator is consistent across years.

For high school students, outcomes include SAT/ACT participation and proficiency, attendance rates, dropout rates, college readiness metrics such as the percentage meeting readiness in Reading, Mathematics, or Both, and AP/IB participation and performance, which are available in increasing detail over time.\footnote{Because of time constraints, I am still working on identifying consistent measures across years for these outcomes, similar to the approach used for SSI.} Demographic controls include the percentage of students who are economically disadvantaged, measured consistently across years as the share of students eligible for free or reduced-price lunch, the same measure used by USAC to determine E-Rate discount rates.

\textbf{Data Limitations:} TEA outcomes are not fully consistent across years due to periodic changes in STAAR assessment reporting, and some subgroup observations remain partially masked even among schools which are larger than the 1A designation, requiring careful treatment of missing values during data construction.\footnote{For future work related to this project, accessing the Texas Education Research Center (ERC) data may be preferable, as it provides more detailed and comprehensive information.} The E-Rate data’s asynchronous approval structure introduces potential temporal mismatches between funding approval and implementation. Additionally, applications that were denied or pending are not visible, and the approval year does not always correspond to the application year, as some requests take multiple years between submission and approval.







\end{document}