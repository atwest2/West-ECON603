\input{preamble}


\begin{document}
\begin{singlespace}
\title{Data Section}
\author{Austyn West\thanks{Department of Economics, Texas A\&M University.}}
\date{October 30, 2025}
\maketitle
\end{singlespace}

\section*{Empirical Design}

This study applies the Difference-in-Discontinuities (DiDisc) framework developed by \citet{grembi2016} to identify the causal effect of the 2014 E-Rate Modernization Order on educational outcomes in Texas public schools. The DiDisc approach combines a regression discontinuity (RD) in program eligibility with a time discontinuity at the start of the policy reform. By differencing out any pre-existing jump in outcomes at each eligibility threshold, the design isolates the reform-induced change in the discontinuity and can therefore identify the local causal effect of the 2014 policy change.

\textbf{Institutional Background.} The 2014 reform substantially altered the structure of E-Rate funding. It raised the annual cap from \$2.4 billion to \$3.9 billion, prioritized high-speed broadband (Category 1) connections, and introduced per-student budgets for internal Wi-Fi (Category 2) services. The maximum subsidy rate for Category 2 was reduced from 90 percent to 85 percent. E-Rate discounts are determined at the district level according to the share of students eligible for the National School Lunch Program (NSLP), generating a series of discrete jumps in funding generosity at eligibility thresholds. Major thresholds occur at 1, 20, 35, 50, and 75 percent NSLP.\footnote{See Universal Service Administrative Company, ``E-Rate Discount Matrix,'' \url{https://www.usac.org/wp-content/uploads/e-rate/documents/samples/Discount-Matrix.pdf}.} For example, districts with between 75 and 100 percent of students eligible for NSLP qualify for the maximum 90 percent and 85 percent discounts on Category 1 and Category 2 services, respectively. Urban and rural distinctions slightly affect the precise discount levels, but the key forcing variable is the NSLP percentage.

\textbf{Identification Strategy.} The design exploits both the discontinuous variation in E-Rate discount rates at these NSLP thresholds and the time variation induced by the 2014 reform. This dual source of variation enables estimation of local causal effects for schools near each eligibility boundary. The baseline analysis centers on the 75 percent threshold, where the reform most likely produced the largest change in funding intensity, while additional analyses examine the 20, 35, and 50 percent cutoffs to assess robustness and potential heterogeneity in treatment effects across the discount matrix. For each cutoff $c$, the specification takes the form
\begin{align}
Y_{ist} 
&= \beta_{0c} + \beta_{1c}\text{Post}_t + \beta_{2c}\text{Above}_{ic} + \beta_{3c}(\text{Post}_t \times \text{Above}_{ic}) \nonumber \\[4pt]
&\quad + f_c(X_{it}) + f_c(X_{it}) \cdot \text{Post}_t + f_c(X_{it}) \cdot \text{Above}_{ic} \nonumber \\[4pt]
&\quad + \mu_s + \tau_t + \varepsilon_{ist}.
\label{eq:didisc}
\end{align}

\textbf{Variable Definitions.} Here, $Y_{ist}$ denotes the outcome of interest—such as the share of students meeting grade-level standards, SAT/ACT participation, dropout rates, or college readiness—for school $s$ in district $i$ and year $t$. The first modernized funding year began disbursements in July 2015, corresponding to the 2015--16 school year; thus, $\text{Post}_t=1$ for 2015--16 onward, reflecting the first full academic year after the reform. Pre-reform years (2012--13 to 2014--15) provide three years of pre-trends. The running variable $X_{it}$ measures the normalized distance of a district’s NSLP percentage from cutoff $c$. The function $f_c(X_{it})$ flexibly captures local trends around each threshold using local linear or low-order polynomial functions. School fixed effects $\mu_s$ control for time-invariant heterogeneity, and year fixed effects $\tau_t$ absorb common shocks. The coefficient $\beta_{3c}$ measures the Difference-in-Discontinuities estimate at cutoff $c$, representing the change in the outcome discontinuity at that threshold between the pre- and post-reform periods.

\textbf{Estimation.} The parameters of equation (\ref{eq:didisc}) are estimated using local linear regressions on either side of each cutoff within an optimal bandwidth, following \citet{calonico2014}. Robustness checks vary bandwidths and polynomial orders and compare results across thresholds to examine consistency of the reform’s effects along the NSLP distribution. Standard errors are clustered at the district level. The design allows for separate pre- and post-reform functional forms and distinct slopes on each side of the threshold, ensuring flexibility in the local fit and mitigating bias from misspecification.

\textbf{Identification Assumptions and Validation.} Identification relies on continuity of potential outcomes at each NSLP threshold in the absence of the reform and on local parallel trends between districts just above and below the cutoff. These assumptions are evaluated through placebo tests using pre-reform data, falsification exercises at non-policy thresholds, and covariate balance checks in the pre-2014 period. Density tests following \citet{mccrary2008} are used to confirm that districts did not manipulate their reported NSLP percentages to cross eligibility thresholds. Finally, pre-trend analyses from 2012--2014 provide further evidence on the validity of the local parallel-trends assumption.


\bibliographystyle{chicago}
\bibliography{Bibliography}



\end{document}